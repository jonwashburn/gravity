\documentclass[12pt]{article}
\usepackage{amsmath,amssymb}
\usepackage{geometry}
\geometry{margin=1in}
\usepackage{hyperref}

\title{Dual Derivation of the \textit{Effective} Recognition Length $\lambda_{\mathrm{eff}}$}
\author{Jonathan Washburn \\ Recognition Physics Institute, Austin TX, USA}
\date{May 14, 2025}

\begin{document}
\maketitle

\begin{abstract}
Recognition Science (RS) fixes a \emph{microscopic} recognition length
\[
\lambda_{\mathrm{micro}}\;=\;\sqrt{\frac{\hbar G}{\pi c^{3}}}=7.23\times10^{-36}\,\text{m},
\]
the radius of the smallest causal diamond able to lock one unit of backlog energy at the
Planck density.  In realistic environments only a vanishing fraction $f\ll1$ of these
Planck--scale pixels is occupied, giving rise to a \emph{mesoscopic} coverage length
\[
\lambda_{\mathrm{eff}}\;=\;\lambda_{\mathrm{micro}}\;f^{-1/4}.
\]
We derive $\lambda_{\mathrm{eff}}$ independently from solar luminosity and from the cosmic
dark--energy density, obtaining a consistent occupancy fraction
$f\simeq3.3\times10^{-122}$ and
$\lambda_{\mathrm{eff}}\approx(60\pm4)\,\mu\text{m}$.  The result reconciles earlier
apparent conflicts without introducing new constants and leaves all curvature--budget
theorems of RS and LNAL intact.
\end{abstract}

\section*{Definitions}
\begin{itemize}
\item \textbf{Microscopic recognition length}
      \[
      \boxed{\lambda_{\mathrm{micro}}\equiv\sqrt{\frac{\hbar G}{\pi c^{3}}}
             =7.23\times10^{-36}\,\text{m}}
      \]
\item \textbf{Occupancy fraction}
      \[
      f\equiv\text{mean fraction of Planck--scale pixels that carry one backlog unit}
      \quad(0<f\ll1).
      \]
\item \textbf{Effective recognition length}
      \[
      \boxed{\lambda_{\mathrm{eff}}\equiv\lambda_{\mathrm{micro}}\,f^{-1/4}}
      \]
\end{itemize}

\section{Stellar--Balance Route}
For an $n=3$ radiation polytrope ($K\simeq20.05$) of mass $M$, radius $R$, and luminosity
$L$, RS backlog is $B=\chi K M^{2}/R^{3}$ with $\chi=\varphi/\pi$.
A photon leaving the star erases one \emph{occupied} area cell $\lambda_{\mathrm{eff}}^{2}$.  Through
an optical depth $\tau$ the drain time is
$\tau_{\mathrm{rec}}=\tau\lambda_{\mathrm{eff}}/c$.  Steady--state balance $B/\tau_{\mathrm{rec}}=L$
yields
\[
\lambda_{\mathrm{eff}}=
\frac{\chi K G c}{\tau}\,
\frac{M^{2}}{L\,R^{3}}.
\]
With solar data $M_\odot,R_\odot,L_\odot$ and $\tau=7.0\times10^{10}$,
\[
\boxed{\lambda_{\mathrm{eff}}^{(\star)}=6.3\times10^{-5}\,\text{m}}
\quad\Longrightarrow\quad
f^{(\star)}=(\lambda_{\mathrm{micro}}/\lambda_{\mathrm{eff}}^{(\star)})^{4}
          \simeq3.2\times10^{-122}.
\]

\section{Vacuum--Energy Route}
RS vacuum backlog density is
$\rho_{\text{vac}}=f\,\chi\hbar c/(2\lambda_{\mathrm{micro}}^{4})$.
Equating to $\rho_{\text{obs}}^{\Lambda}=6.0\times10^{-27}\,\text{kg\,m}^{-3}$ gives
\[
f=
\frac{2\lambda_{\mathrm{micro}}^{4}\rho_{\text{obs}}^{\Lambda}}{\chi\hbar c},
\qquad
\lambda_{\mathrm{eff}}^{(\Lambda)}=
\lambda_{\mathrm{micro}}\,f^{-1/4}
      =\left(\frac{\chi\hbar c}{2\rho_{\text{obs}}^{\Lambda}}\right)^{1/4}.
\]
Numerically,
\[
\boxed{\lambda_{\mathrm{eff}}^{(\Lambda)}=5.9\times10^{-5}\,\text{m}}
\quad\Longrightarrow\quad
f^{(\Lambda)}\simeq3.4\times10^{-122}.
\]

\section{Concordance}
The two derivations agree within $7\%$ and pin a common occupancy factor
\[
f=(3.3\pm0.3)\times10^{-122},
\qquad
\lambda_{\mathrm{eff}}\approx(60\pm4)\,\mu\text{m}.
\]
No conflict exists: $\lambda_{\mathrm{micro}}$ remains the universal Planck--scale pixel,
while $\lambda_{\mathrm{eff}}$ is an emergent lattice constant set by sparsity.

\section{Significance}
\begin{itemize}
\item All curvature--budget, token--parity, and $\pm4$ ladder proofs in LNAL rely only on
      $\lambda_{\mathrm{micro}}$ and remain unchanged.
\item $\lambda_{\mathrm{eff}}\sim60\,\mu\text{m}$ calibrates laboratory tests
      (φ--comb gaps, inert--gas Kerr null, etc.) and provides a concrete design target.
\item Improved measurements of solar opacity or $\rho_{\text{obs}}^{\Lambda}$ will refine
      $f$ and challenge RS at the sub-percent level.
\end{itemize}

\section*{Acknowledgements}
The author thanks colleagues at the Recognition Physics Institute for clarifying discussions
on pixel sparsity and curvature budgeting.

\bibliographystyle{unsrt}
\begin{thebibliography}{9}
\bibitem{axioms}
J.~Washburn, \textit{Foundational Axioms of Recognition Science and a Proof of
Consistent Existence} (2025).
\end{thebibliography}

\end{document}
